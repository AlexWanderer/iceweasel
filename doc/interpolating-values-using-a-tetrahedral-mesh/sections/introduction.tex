\section{Introduction}

A computer game featuring localised gravity -- the ``down'' direction of objects
changes  depending on where you are located  in  3D  space  --  is  achieved  by
creating a descrete  vector  field. Normalised vectors, referred to as ``Gravity
Vectors'', are placed at key locations  in  the 3D world to define the direction
gravity should have at that location.

Ideally, a continuous vector field is desired,  as that would remove the need to
smooth out  sudden  changes in gravity.

The  gravity  vector locations are used to  construct  a  tetrahedral  mesh.  An
arbitrary 3D location, if located inside any  one  of  the  tetrahedrons, can be
converted into barycentric coordinates and used  to  calculate a new directional
vector based on the interpolated values of the four  gravity vectors forming the
tetrhahedron.

If the 3D location is located outside the convex hull of the mesh, it is unclear
how  the  vector  field should be  extrapolated.  To  address  this,  we  create
``infinite tetrahedrons'' using the faces of the convex hull. The result is such
that a 3D  point located outside of the convex hull is projected either onto one
of the faces on the convex  hull  (if  the tetrahedron has one vertex located in
infinity),  projected  onto  one  of the  edges  on  the  convex  hull  (if  the
tetrahedron has two vertices  located  in infinity) or projected onto one of the
vertices  of  the  convex hull (if the tetrahedron has three vertices located in
infinity). The projected location can be converted to barycentric coordinates as
usual and the interpolated vector can be computed.

The  result is a continuous vector field,  capable  of  mapping  any  finite  3D
location to a gravitational vector.
