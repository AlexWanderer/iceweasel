\section{Theory}

Four   different  cases  need  to  be  distinguished:  Normal  tetrahedrons,   a
tetrahedron with one vertex located infinitely far away, a  tetrahedron with two
vertices  located  infinitely  far  away,  and a tetrahedron with three vertices
located infinitely away.

The barycentric coordinates of a 3D point in space  need  to be computed for all
cases. Additionally, a method  for  checking  boundaries  needs to exist for all
cases. Two 4x4 transformation  matrices  are  proposed  to  generalise these two
requirements. The construction of these matrices is different in every case, and
is described in detail in the following sections.

\subsection{Transformation Matrices}

\subsection{Transforming from Cartesian to Barycentric}

A 3D tetrahedron,  a  polyhedron having four triangular faces and four vertices,
is defined  by its four vertices $v_1$, $v_2$, $v_3$ and $v_4$, where $v_n$ is a
3 dimensional point in Cartesian space $v_n = \begin{bmatrix} x_n & y_n & z_n \\
\end{bmatrix}^T$.  The  barycentric  coordinates  are defined so that the  first
vertex $r_1$ maps to barycentric coordinates $\lambda_1 = \begin{bmatrix} 1  & 0
&  0 & 0 \\ \end{bmatrix}$,  $r_2  \to  \begin{bmatrix}  0  &  1  &  0  &  0  \\
\end{bmatrix}$, etc. and that the sum of barycentric parameters $\sum\lambda_n
= 1$.

This is a linear transformation and the problem can be written in matrix form so
that $\vec{v} = B \inverse \vec{\lambda}$ with $B \inverse = \begin{bmatrix} v_1
|  v_2  |  v_3  |  v_4  \\  \end{bmatrix}$  and $\vec{\lambda} = \begin{bmatrix}
\lambda_1 &  \lambda_2 & \lambda_3 \\ \end{bmatrix}^T$. The condition $\lambda_1
+  \lambda_2  +  \lambda_3 + \lambda_4 = 1$ can be augmented into the matrix  to
form the final equation:

\begin{align*}
    \begin{bmatrix}
        v_{1_x} & v_{2_x} & v_{3_x} & v_{4_x} \\
        v_{1_y} & v_{2_y} & v_{3_y} & v_{4_y} \\
        v_{1_z} & v_{2_z} & v_{3_z} & v_{4_z} \\
        1 & 1 & 1 & 1 \\
    \end{bmatrix}
    \vec{\lambda} &= \begin{bmatrix}
        x \\
        y \\
        z \\
        1 \\
    \end{bmatrix} \\
\end{align*}

Where $x$, $y$,  and  $z$  define a 3D point in Cartesian space. The barycentric
coordinates $\lambda$ can be obtained  by  solving  this  linear  equation, thus
resulting in the transformation matrix $B$:

\begin{align}
    \label{eq:bary:cartesian}
    B = \begin{bmatrix}
        v_{1_x} & v_{2_x} & v_{3_x} & v_{4_x} \\
        v_{1_y} & v_{2_y} & v_{3_y} & v_{4_y} \\
        v_{1_z} & v_{2_z} & v_{3_z} & v_{4_z} \\
        1 & 1 & 1 & 1 \\
    \end{bmatrix}\inverse
\end{align}


\subsubsection{General Projection}

For all other cases where the tetrahedron has infinite vertices, we will need to
construct a projection matrix.

Given a subspace $V = \text{span} \begin{bmatrix} \vec{e_1} | \vec{e_2} | \ldots
\end{bmatrix}$ where $\vec{e_n}$ is the basis for $V$  and  $\vec{x} , \vec{e_n}
\in  \mathbb{R}^N$,  the  projection  of  $\vec{x}$  onto  $V$  is  defined  as:

\begin{align}
    \label{eq:projection_origin}
    \text{proj}_V\vec{x} = A\left(A\transpose A\right)\inverse A\transpose\vec{x}
\end{align}

This projection is only valid for $\vec{v_1} = \vec{0}$.  If the triangle has an
offset in 3D space,  then  the  offset  must be subtracted before performing the
projection and added back after the projection. A translation matrix is used:

\begin{align}
    \label{eq:translate_offset}
    T = \begin{bmatrix}
        1 & 0 & 0 & -v_{1_x} \\
        0 & 1 & 0 & -v_{1_y} \\
        0 & 0 & 1 & -v_{1_z} \\
        0 & 0 & 0 & 1 \\
    \end{bmatrix}
\end{align}

The final, general projection matrix can be constructed with:

\begin{align}
    \label{eq:projection}
    \text{proj}_V\vec{x} = TA\left(A\transpose A\right)\inverse A\transpose T\inverse\vec{x}
\end{align}

\subsubsection{Barycentric Coordinates of a Face Projection}

In the case of a single vertex of the tetrahedron being  located  infinitely far
away, any 3D point located inside its volume will be projected onto the triangle
formed  by the three finite vertex locations. Thus, a projection matrix must  be
constructed.

The face of a  tetrahedron  is  defined  by  the three vertices $v_1$, $v_2$ and
$v_3$,  where  $v_n \in \mathbb{R}^3$. The 3  dimensional  Cartesian  coordinate
$\vec{x}\in\mathbb{R}^3$  is  projected  onto one of the tetrahedron's triangles
using equation \ref{eq:projection} where $\vec{a}  =  \vec{v_2}  - \vec{v_1}$
and  $\vec{b} = \vec{v_3}  -  \vec{v_1}$  and  the  matrix  $A$  is  defined  as
$A=\begin{bmatrix} a|b \end{bmatrix}$.

The matrices $B$ and $\text{proj}_V$ from equations \ref{eq:bary:cartesian}  and
\ref{eq:projection} may be combined to form the final matrix  for transforming a
3D coordinate $\vec{x}$ into projected
barycentric coordinates $\vec{\lambda}$:

\begin{align}
    \label{eq:bary:triangle}
    \vec{\lambda} = TA\left(A\transpose A\right)\inverse A\transpose T\inverse B \vec{x}
\end{align}

In  an effort to be more verbose, equation \ref{eq:bary:triangle} is broken down
and   constructed   step-by-step   using  the  triangle  vertices   $\vec{v_1}$,
$\vec{v_2}$, $\vec{v_3}$.

\begin{align*}
    \vec{a} &= \vec{v_2} - \vec{v_1} \\
    \vec{b} &= \vec{v_3} - \vec{v_1} \\
    \text{proj}_V &= A\left(A\transpose A\right)\inverse A\transpose \\
    \begin{bmatrix}
        m_{00} & m_{01} & m_{02} \\
        m_{10} & m_{11} & m_{12} \\
        m_{20} & m_{21} & m_{22} \\
    \end{bmatrix}
    &=  \begin{bmatrix}
            a_x & b_x \\
            a_y & b_y \\
            a_z & b_z \\
        \end{bmatrix}
        \left(
            \begin{bmatrix}
                a_x & a_y & a_z \\
                b_x & b_y & b_z \\
            \end{bmatrix}
            \begin{bmatrix}
                a_x & b_x \\
                a_y & b_y \\
                a_z & b_z \\
            \end{bmatrix}
        \right)\inverse
        \begin{bmatrix}
            a_x & a_y & a_z \\
            b_x & b_y & b_z \\
        \end{bmatrix}
\end{align*}

The  calculated  projection  matrix  can be sandwiched between  the  translation
matrix $T$ from equation \ref{eq:translate_offset} and then be  multiplied  with
the  barycentric transformation matrix $B$ from equation \ref{eq:bary:cartesian}
to yield the final transformation matrix.

\begin{align*}
    \vec{\lambda} = 
    \begin{bmatrix}
        1 & 0 & 0 & -v_{1_x} \\
        0 & 1 & 0 & -v_{1_y} \\
        0 & 0 & 1 & -v_{1_z} \\
        0 & 0 & 0 & 1 \\
    \end{bmatrix}
    \begin{bmatrix}
        m_{00} & m_{01} & m_{02} & 0 \\
        m_{10} & m_{11} & m_{12} & 0 \\
        m_{20} & m_{21} & m_{22} & 0 \\
        0      & 0      & 0      & 1 \\
    \end{bmatrix}
    \begin{bmatrix}
        1 & 0 & 0 & -v_{1_x} \\
        0 & 1 & 0 & -v_{1_y} \\
        0 & 0 & 1 & -v_{1_z} \\
        0 & 0 & 0 & 1 \\
    \end{bmatrix}\inverse
    \begin{bmatrix}
        v_{1_x} & v_{2_x} & v_{3_x} & v_{4_x} \\
        v_{1_y} & v_{2_y} & v_{3_y} & v_{4_y} \\
        v_{1_z} & v_{2_z} & v_{3_z} & v_{4_z} \\
        1 & 1 & 1 & 1 \\
    \end{bmatrix}
    \vec{x}
\end{align*}


\subsubsection{Barycentric Coordinates of an Edge Projection}

In the case of two vertices of the  tetrahedron  being  located  infinitely  far
away, any 3D point located inside its volume will  be  projected  onto  the edge
formed by the two  finite  vertex  locations.  Thus, a projection matrix must be
constructed.

The edge of  a tetrahedron is defined by the two vertices $v_1$ and $v_2$, where
$v_n    \in    \mathbb{R}^3$.    The   3   dimensional   Cartesian    coordinate
$\vec{x}\in\mathbb{R}^3$  is projected onto one of the tetrahedron's edges using
equation  \ref{eq:projection}  where  the  matrix  $A$   is   defined  as  $A  =
\begin{bmatrix} \vec{v_2} - \vec{v_1} \end{bmatrix}$.

The matrices $B$ and $\text{proj}_V$ from equations \ref{eq:bary:cartesian}  and
\ref{eq:projection} may be combined to form the final matrix  for transforming a
3D coordinate $\vec{x}$ into projected
barycentric coordinates $\vec{\lambda}$:

\begin{align}
    \label{eq:bary:edge}
    \vec{\lambda} = TA\left(A\transpose A\right)\inverse A\transpose T\inverse B \vec{x}
\end{align}

In an effort to be more verbose, equation \ref{eq:bary:edge} is broken down  and
constructed  step-by-step  using  the  edge  vertices  $\vec{v_1}$, $\vec{v_2}$.

\begin{align*}
    \vec{a} &= \vec{v_2} - \vec{v_1} \\
    \text{proj}_V &= A\left(A\transpose A\right)\inverse A\transpose \\
    \begin{bmatrix}
        m_{00} & m_{01} & m_{02} \\
        m_{10} & m_{11} & m_{12} \\
        m_{20} & m_{21} & m_{22} \\
    \end{bmatrix}
    &=  \begin{bmatrix}
            a_x \\
            a_y \\
            a_z \\
        \end{bmatrix}
        \left(
            \begin{bmatrix}
                a_x & a_y & a_z \\
            \end{bmatrix}
            \begin{bmatrix}
                a_x \\
                a_y \\
                a_z \\
            \end{bmatrix}
        \right)\inverse
        \begin{bmatrix}
            a_x & a_y & a_z \\
        \end{bmatrix}
\end{align*}

The  calculated  projection matrix can be  sandwiched  between  the  translation
matrix  $T$  from equation \ref{eq:translate_offset} and then be multiplied with
the  barycentric transformation matrix $B$ from equation \ref{eq:bary:cartesian}
to yield the final transformation matrix.

\begin{align*}
    \vec{\lambda} = 
    \begin{bmatrix}
        1 & 0 & 0 & -v_{1_x} \\
        0 & 1 & 0 & -v_{1_y} \\
        0 & 0 & 1 & -v_{1_z} \\
        0 & 0 & 0 & 1 \\
    \end{bmatrix}
    \begin{bmatrix}
        m_{00} & m_{01} & m_{02} & 0 \\
        m_{10} & m_{11} & m_{12} & 0 \\
        m_{20} & m_{21} & m_{22} & 0 \\
        0      & 0      & 0      & 1 \\
    \end{bmatrix}
    \begin{bmatrix}
        1 & 0 & 0 & -v_{1_x} \\
        0 & 1 & 0 & -v_{1_y} \\
        0 & 0 & 1 & -v_{1_z} \\
        0 & 0 & 0 & 1 \\
    \end{bmatrix}\inverse
    \begin{bmatrix}
        v_{1_x} & v_{2_x} & v_{3_x} & v_{4_x} \\
        v_{1_y} & v_{2_y} & v_{3_y} & v_{4_y} \\
        v_{1_z} & v_{2_z} & v_{3_z} & v_{4_z} \\
        1 & 1 & 1 & 1 \\
    \end{bmatrix}
    \vec{x}
\end{align*}


\subsubsection{Barycentric Coordinates of a Point Projection}

In the case of three vertices of  the  tetrahedron  being located infinitely far
away, any 3D point located  inside its volume will effectively be projected onto
the one finite vertex.

The  3  dimensional Cartesian coordinate $\vec{x}\in\mathbb{R}^3$  is  projected
onto  one  of  the  tetrahedron's  vertices  $v_1 \in  \mathbb{R}^3$  using  the
following projection matrix:

\begin{equation}
    \label{eq:point}
    P = \begin{bmatrix}
        0 & 0 & 0 & v_{1_x} \\
        0 & 0 & 0 & v_{1_y} \\
        0 & 0 & 0 & v_{1_z} \\
        0 & 0 & 0 & 1 \\
    \end{bmatrix}
\end{equation}

The barycentric coordinates $\vec{\lambda}$  of  $\vec{x}$  can  be  computed by
combining   matrices   $P$   and   $B$   from   equations   \ref{eq:point}   and
\ref{eq:bary:cartesian}:

\begin{equation}
    \vec{\lambda} = PB\vec{x}
\end{equation}

It should be noted that a projection  matrix may be overkill for this particular
case. However, it is necessary if one wants to generalise  the implementation by
using a 4x4 matrix.




\subsection{Boundary Check}

In this  section  we  discuss  how  to  determine if a point is located inside a
tetrahedron or not. This includes cases  where  vertices  of the tetrahedron are
located infinitely far away.

The easist case is  a  normal  tetrahedron with no infinite vertices. A 3D point
$\vec{x}$  is  located  inside  the  tetrahedron  if its barycentric coordinates
$\vec{\lambda}  =  \begin{bmatrix}  \lambda_1,  \lambda_2, \lambda_3,  \lambda_4
\end{bmatrix} \transpose$ satisfy the condition:

\begin{equation}
    \label{eq:boundary_check}
    0\le\lambda_n\le 1
\end{equation}

If one vertex is  located  in  infinity  then  one of the barycentric parameters
$\lambda_1$ will always equal 0. Therefore, we need  to additionally check which
side of  the  triangle, defined by $\vec{v_1}$, $\vec{v_2}$, and $\vec{v_3}$ the
point  $\vec{x}$ is being  projected  from.  The  following  condition  must  be
satisfied   in   addition    to    the    condition    described   in   equation
\ref{eq:boundary_check}

\begin{align}
    \left(\vec{v_2}-\vec{v_1}\right)\times\left(\vec{v_3}-\vec{v_1}\right) \odot \left(\vec{x}-\vec{v_1}\right) \ge 0
\end{align}

If  two vertices are located in infinity then two of the barycentric  parameters
$\lambda_{1,2}$ will always equal 0.


\section{Another Approach}

Instead of dealing with four different cases and having to build eight different
transformation matrices,  another approach is to cast a ray from the 3D location
$\vec{x}$ to the center of the convex hull of the tetrahedral mesh and determine
the location of intercection on the hull's surface.

\subsection{Intersection of a Ray and a Triangle}

A  triangle  is  defined  by  its  three  vertices  $\vec{v_1}$, $\vec{v_2}$ and
$\vec{v_3}$. A point $\vec{T}(\lambda_1,\lambda_2)$ on a  triangle  is  given by

\begin{equation}
    \label{eq:point_on_triangle}
    \vec{T}(\lambda_1,\lambda_2) = (1-\lambda_1-\lambda_2)\vec{v_1} + u\vec{v_2} + v\vec{v_3}
\end{equation}

Where  $(\lambda_1,\lambda_2)$  are the barycentric coordinates of the triangle,
which must fulfill the requirement $0\le\lambda_n\le 1$.

A  ray  $\vec{R}(t)$  with  origin $\vec{r_0}$ and normalised direction  $\vec{d}$  is
defined as

\begin{equation}
    \label{eq:ray}
    \vec{R}(t) = \vec{r_0} + t\vec{d}
\end{equation}

Computing  the  intersection  between  the  ray  $\vec{R}(t)$  and  the triangle
$\vec{T}  (\lambda_1,  \lambda_2)$  is  equivalent  to  $\vec{R}(t)  =   \vec{T}
(\lambda_1, \lambda_2)$, which yields

\begin{equation}
    \vec{r_0} + t\vec{d} = (1-\lambda_1-\lambda_2)\vec{v_1} + \lambda_1\vec{v_2} + \lambda_2\vec{v_3}
\end{equation}

Rearranging the terms gives

\begin{equation}
    \label{eq:ray_equation}
    \begin{bmatrix} -d & \vec{v_2}-\vec{v_1} & \vec{v_3}-\vec{v_1} \\ \end{bmatrix}
    \begin{bmatrix} t \\ \lambda_1 \\ \lambda_2 \\ \end{bmatrix} = \vec{r_0} - \vec{v_1}
\end{equation}

This means the barycentric coordinates $(\lambda_1, \lambda_2)$ and the distance
$t$ from the ray origin to the intersection point can be  found  by  solving the
lienar system of equations above.

Denoting $\vec{a} = \vec{v_2} -  \vec{v_1}$,  $\vec{b}  = \vec{v_3} - \vec{v_1}$
and   $\vec{c}   =   \vec{r_0}   -   \vec{v_1}$,   the   solution  to   equation
\ref{eq:ray_equation} is obtained by using Cramer's rule:

\begin{equation}
    \label{eq:ray_equation_solved}
    \begin{bmatrix} t \\ \lambda_1 \\ \lambda_2 \\ \end{bmatrix}
    = \frac{1}{\begin{vmatrix} -\vec{d} & \vec{a} & \vec{b} \end{vmatrix}}
    \begin{bmatrix}\begin{vmatrix}
        \vec{c}  & \vec{a} & \vec{b} \\
        -\vec{d} & \vec{c} & \vec{b} \\
        -\vec{d} & \vec{a} & \vec{c} \\
    \end{vmatrix}\end{bmatrix}
\end{equation}

From linear algebra, we  know  that $\begin{vmatrix} \vec{A} & \vec{B} & \vec{C}
\\ \end{vmatrix} =  -\left(  \vec{A}  \times  \vec{C}  \right) \cdot B = -\left(
\vec{C} \times \vec{B} \right) \cdot A$.  Equation  \ref{eq:ray_equation_solved}
could therefore be rewritten as

\begin{equation}
    \begin{bmatrix} t \\ \lambda_1 \\ \lambda_2 \\ \end{bmatrix} =
    \frac{1}{\left(\vec{d}\times\vec{b}\right)\cdot\vec{a}}
    \begin{bmatrix}
        \left(\vec{c}\times\vec{a}\right)\cdot\vec{b} \\
        \left(\vec{d}\times\vec{b}\right)\cdot\vec{c} \\
        \left(\vec{c}\times\vec{a}\right)\cdot\vec{d} \\
    \end{bmatrix}
\end{equation}


