\subsubsection{Barycentric Coordinates of a Point Projection}

In the case of three vertices of  the  tetrahedron  being located infinitely far
away, any 3D point located  inside its volume will effectively be projected onto
the one finite vertex.

The  3  dimensional Cartesian coordinate $\vec{x}\in\mathbb{R}^3$  is  projected
onto  one  of  the  tetrahedron's  vertices  $v_1 \in  \mathbb{R}^3$  using  the
following projection matrix:

\begin{equation}
    \label{eq:point}
    P = \begin{bmatrix}
        0 & 0 & 0 & v_{1_x} \\
        0 & 0 & 0 & v_{1_y} \\
        0 & 0 & 0 & v_{1_z} \\
        0 & 0 & 0 & 1 \\
    \end{bmatrix}
\end{equation}

The barycentric coordinates $\vec{\lambda}$  of  $\vec{x}$  can  be  computed by
combining   matrices   $P$   and   $B$   from   equations   \ref{eq:point}   and
\ref{eq:bary:cartesian}:

\begin{equation}
    \vec{\lambda} = PB\vec{x}
\end{equation}

It should be noted that a projection  matrix may be overkill for this particular
case. However, it is necessary if one wants to generalise  the implementation by
using a 4x4 matrix.

