\subsubsection{Barycentric Coordinates of a Face Projection}

In the case of a single vertex of the tetrahedron being  located  infinitely far
away, any 3D point located inside its volume will be projected onto the triangle
formed  by the three finite vertex locations. Thus, a projection matrix must  be
constructed.

The face of a  tetrahedron  is  defined  by  the three vertices $v_1$, $v_2$ and
$v_3$,  where  $v_n \in \mathbb{R}^3$. The 3  dimensional  Cartesian  coordinate
$\vec{x}\in\mathbb{R}^3$  is  projected  onto one of the tetrahedron's triangles
using equation \ref{eq:projection} where $\vec{a}  =  \vec{v_2}  - \vec{v_1}$
and  $\vec{b} = \vec{v_3}  -  \vec{v_1}$  and  the  matrix  $A$  is  defined  as
$A=\begin{bmatrix} a|b \end{bmatrix}$.

The matrices $B$ and $\text{proj}_V$ from equations \ref{eq:bary:cartesian}  and
\ref{eq:projection} may be combined to form the final matrix  for transforming a
3D coordinate $\vec{x}$ into projected
barycentric coordinates $\vec{\lambda}$:

\begin{align}
    \label{eq:bary:triangle}
    \vec{\lambda} = TA\left(A\transpose A\right)\inverse A\transpose T\inverse B \vec{x}
\end{align}

In  an effort to be more verbose, equation \ref{eq:bary:triangle} is broken down
and   constructed   step-by-step   using  the  triangle  vertices   $\vec{v_1}$,
$\vec{v_2}$, $\vec{v_3}$.

\begin{align*}
    \vec{a} &= \vec{v_2} - \vec{v_1} \\
    \vec{b} &= \vec{v_3} - \vec{v_1} \\
    \text{proj}_V &= A\left(A\transpose A\right)\inverse A\transpose \\
    \begin{bmatrix}
        m_{00} & m_{01} & m_{02} \\
        m_{10} & m_{11} & m_{12} \\
        m_{20} & m_{21} & m_{22} \\
    \end{bmatrix}
    &=  \begin{bmatrix}
            a_x & b_x \\
            a_y & b_y \\
            a_z & b_z \\
        \end{bmatrix}
        \left(
            \begin{bmatrix}
                a_x & a_y & a_z \\
                b_x & b_y & b_z \\
            \end{bmatrix}
            \begin{bmatrix}
                a_x & b_x \\
                a_y & b_y \\
                a_z & b_z \\
            \end{bmatrix}
        \right)\inverse
        \begin{bmatrix}
            a_x & a_y & a_z \\
            b_x & b_y & b_z \\
        \end{bmatrix}
\end{align*}

The  calculated  projection  matrix  can be sandwiched between  the  translation
matrix $T$ from equation \ref{eq:translate_offset} and then be  multiplied  with
the  barycentric transformation matrix $B$ from equation \ref{eq:bary:cartesian}
to yield the final transformation matrix.

\begin{align*}
    \vec{\lambda} = 
    \begin{bmatrix}
        1 & 0 & 0 & -v_{1_x} \\
        0 & 1 & 0 & -v_{1_y} \\
        0 & 0 & 1 & -v_{1_z} \\
        0 & 0 & 0 & 1 \\
    \end{bmatrix}
    \begin{bmatrix}
        m_{00} & m_{01} & m_{02} & 0 \\
        m_{10} & m_{11} & m_{12} & 0 \\
        m_{20} & m_{21} & m_{22} & 0 \\
        0      & 0      & 0      & 1 \\
    \end{bmatrix}
    \begin{bmatrix}
        1 & 0 & 0 & -v_{1_x} \\
        0 & 1 & 0 & -v_{1_y} \\
        0 & 0 & 1 & -v_{1_z} \\
        0 & 0 & 0 & 1 \\
    \end{bmatrix}\inverse
    \begin{bmatrix}
        v_{1_x} & v_{2_x} & v_{3_x} & v_{4_x} \\
        v_{1_y} & v_{2_y} & v_{3_y} & v_{4_y} \\
        v_{1_z} & v_{2_z} & v_{3_z} & v_{4_z} \\
        1 & 1 & 1 & 1 \\
    \end{bmatrix}
    \vec{x}
\end{align*}

