\section{Projection Matrices}

\subsection{Transforming from Cartesian to Barycentric}

A 3D tetrahedron,  a  polyhedron having four triangular faces and four vertices,
is defined  by its four vertices $v_1$, $v_2$, $v_3$ and $v_4$, where $v_n$ is a
3 dimensional point in Cartesian space $v_n = \begin{bmatrix} x_n & y_n & z_n \\
\end{bmatrix}^T$.  The  barycentric  coordinates  are defined so that the  first
vertex $r_1$ maps to barycentric coordinates $\lambda_1 = \begin{bmatrix} 1  & 0
&  0 & 0 \\ \end{bmatrix}$,  $r_2  \to  \begin{bmatrix}  0  &  1  &  0  &  0  \\
\end{bmatrix}$, etc. and that the sum of barycentric parameters $\sum\lambda_n
= 1$.

This is a linear transformation and the problem can be written in matrix form so
that  $\vec{v}  = B\vec{\lambda}$  with  $B=\begin{bmatrix}  v_1|v_2|v_3|v_4  \\
\end{bmatrix}$  and  $\vec{\lambda}  =  \begin{bmatrix} \lambda_1 & \lambda_2  &
\lambda_3 \\ \end{bmatrix}^T$.  The condition $\lambda_1 + \lambda_2 + \lambda_3
+  \lambda_4 = 1$ can be augmented into the matrix to form the  final  equation:

\begin{align}
    \label{eq:bary:cartesian}
    B\vec{\lambda} &= \vec{v} \\
    \begin{bmatrix}
        v_{1_x} & v_{2_x} & v_{3_x} & v_{4_x} \\
        v_{1_y} & v_{2_y} & v_{3_y} & v_{4_y} \\
        v_{1_z} & v_{2_z} & v_{3_z} & v_{4_z} \\
        1 & 1 & 1 & 1 \\
    \end{bmatrix}
    \vec{\lambda} &= \begin{bmatrix}
        x \\
        y \\
        z \\
        1 \\
    \end{bmatrix} \\
\end{align}

Where  $x$,  $y$, and $z$ define a 3D point in Cartesian space.  The  barymetric
coordinates  $\lambda$   can   be   obtained  by  solving  the  linear  equation
\ref{eq:bary:cartesian}.

\subsection{Barycentric Coordinates of a Surface Projection}

Given a subspace $V = \text{span} \begin{bmatrix} \vec{e_1} | \vec{e_2} | \ldots
\end{bmatrix}$ where $\vec{e_n}$ is the basis for $V$  and  $\vec{x} , \vec{e_n}
\in  \mathbb{R}^N$,  the  projection  of  $\vec{x}$  onto  $V$  is  defined  as:

\begin{align}
    \label{eq:triangle_proj}
    \text{proj}_V\vec{x} = A\left(A\transpose A\right)\inverse A\transpose\vec{x}
\end{align}

The face of a  tetrahedron  is  defined  by  the three vertices $v_1$, $v_2$ and
$v_3$,  where  $v_n \in \mathbb{R}^3$. The 3  dimensional  cartesian  coordinate
$\vec{x}\in\mathbb{R}^3$  is  projected  into one of the tetrahedron's triangles
using equation \ref{eq:triangle_proj} where $\vec{a}  =  \vec{v_2}  - \vec{v_1}$
and  $\vec{b} = \vec{v_3}  -  \vec{v_1}$  and  the  matrix  $A$  is  defined  as
$A=\begin{bmatrix} a|b \end{bmatrix}$.

This projection is only valid for $\vec{v_1} = \vec{0}$.  If the triangle has an
offset in 3D space,  then  the  offset  must be subtracted before performing the
projection and added back after the projection. A translation matrix is used:

\begin{align}
    \label{eq:translate_offset}
    T = \begin{bmatrix}
        1 & 0 & 0 & -v_{1_x} \\
        0 & 1 & 0 & -v_{1_y} \\
        0 & 0 & 1 & -v_{1_z} \\
        0 & 0 & 0 & 1 \\
    \end{bmatrix}
\end{align}

The final matrix for transforming  a  3D  coordinate  $\vec{x}$  into  projected
barycentric coordinates can now be constructed  by  combining  the matrices $B$,
$T$ and $A$ from  equations  \ref{eq:bary:cartesian}, \ref{eq:triangle_proj} and
\ref{eq:translate_offset}:

\begin{align}
    \label{eq:bary:triangle}
    \vec{\lambda} = TA\left(A\transpose A\right)\inverse A\transpose T\inverse B \vec{x}
\end{align}

In  an effort to be more verbose, equation \ref{eq:bary:triangle} is broken down
and   constructed   step-by-step   using  the  triangle  vertices   $\vec{v_1}$,
$\vec{v_2}$, $\vec{v_3}$.

\begin{align*}
    \vec{a} &= \vec{v_2} - \vec{v_1} \\
    \vec{b} &= \vec{v_3} - \vec{v_1} \\
    \text{proj}_V\vec{x} &= A\left(A\transpose A\right)\inverse A\transpose \\
    \begin{bmatrix}
        m_{00} & m_{01} & m_{02} \\
        m_{10} & m_{11} & m_{12} \\
        m_{20} & m_{21} & m_{22} \\
    \end{bmatrix}
    &=  \begin{bmatrix}
            a_x & b_x \\
            a_y & b_y \\
            a_z & b_z \\
        \end{bmatrix}
        \left(
            \begin{bmatrix}
                a_x & a_y & a_z \\
                b_x & b_y & b_z \\
            \end{bmatrix}
            \begin{bmatrix}
                a_x & b_x \\
                a_y & b_y \\
                a_z & b_z \\
            \end{bmatrix}
        \right)\inverse
        \begin{bmatrix}
            a_x & a_y & a_z \\
            b_x & b_y & b_z \\
        \end{bmatrix} \\
\end{align*}

The translation matrix from equation \ref{eq:translate_offset} can be applied to
the projection matrix after expanding it to the same dimension:

\begin{align*}
    \vec{\lambda} = 
    \begin{bmatrix}
        1 & 0 & 0 & -v_{1_x} \\
        0 & 1 & 0 & -v_{1_y} \\
        0 & 0 & 1 & -v_{1_z} \\
        0 & 0 & 0 & 1 \\
    \end{bmatrix}
    \begin{bmatrix}
        m_{00} & m_{01} & m_{02} & 0 \\
        m_{10} & m_{11} & m_{12} & 0 \\
        m_{20} & m_{21} & m_{22} & 0 \\
        0      & 0      & 0      & 1 \\
    \end{bmatrix}
    \begin{bmatrix}
        1 & 0 & 0 & -v_{1_x} \\
        0 & 1 & 0 & -v_{1_y} \\
        0 & 0 & 1 & -v_{1_z} \\
        0 & 0 & 0 & 1 \\
    \end{bmatrix}\inverse
    \begin{bmatrix}
        v_{1_x} & v_{2_x} & v_{3_x} & v_{4_x} \\
        v_{1_y} & v_{2_y} & v_{3_y} & v_{4_y} \\
        v_{1_z} & v_{2_z} & v_{3_z} & v_{4_z} \\
        1 & 1 & 1 & 1 \\
    \end{bmatrix}
    \vec{x}
\end{align*}

